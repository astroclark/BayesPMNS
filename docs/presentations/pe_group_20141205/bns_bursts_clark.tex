\documentclass{beamer}
\setbeamertemplate{navigation symbols}{}

\usepackage{beamerthemeshadow}
\usepackage{comment}
\usepackage{epstopdf}
\usepackage{graphicx}
\setbeamertemplate{caption}[numbered]

\hypersetup{colorlinks}

\def\gw#1{gravitational wave#1 (GW#1)\gdef\gw{GW}}
\def\ns#1{neutron star#1 (NS#1)\gdef\ns{NS}}

\newcommand{\red}[1]{{\color{red}{#1}}}

\begin{document}
\setbeamertemplate{caption}{\raggedright\insertcaption\par}

%\title{BNS Windowing Examples}
%\author{James A. Clark}
%\institute{Georgia Institute Of Technology}
%\date{} 

%\begin{frame}[plain]
%\titlepage
%\end{frame}

%\begin{frame}\frametitle{Table of contents}\tableofcontents
%\end{frame} 

%\section{{\tt LALSimulation} \& Burst Injections}

\begin{frame}
    \frametitle{BNS Burst PE}
    NS group search plan:
    \begin{itemize}
        \item Use CWB \& burst PE tools for high-freq. BNS follow-ups
        \item Goal: gather evidence for, and measure peak frequency of
            possible post-merger NS oscillations
    \end{itemize}

    \begin{columns}[]

        \column{0.5\textwidth}

        \begin{center}
            \vspace{-0.5cm}
            \begin{figure}
                %\begin{centering}
                \scalebox{0.3}{
                    \includegraphics{dd2_135135_example.eps}
                }
                %\caption{\tiny Example post-merger waveform spectrum 
                %$    red/magenta=max. match sine-Gaussian.}
                %\end{centering}
            \end{figure}
        \end{center}

        \column{0.5\textwidth}

        \begin{itemize}
            \item Potentially good frequency recovery with sine-Gaussians in LIB
                (see left figure)
            \item Challenge: recover peak frequency without low-freq.
                `confusion'
            \item Investigations underway/to begin to characterise
                reconstruction with CWB / BayesWave
        \end{itemize}

    \end{columns}

\end{frame}

\end{document}

